本研究で開発した2脚ロボットの外観をFig.\ref{robot_outline}に示す.
ロボットの大きさは,高さ510*幅230*奥行き140[mm],重量2.5[kg]であり,股関節にロールとピッチ方向,膝関節と距腿関節にピッチ方向の自由度をもつ.
股関節の2自由度は,ROBOTIS社製Dynamixel xのサーボモータによって駆動され,膝関節と距腿関節は,マッキベン型空気圧人工筋(以下,空気圧人工筋と呼ぶ)によって駆動される.
また,股関節と膝関節,膝関節と距腿関節の間隔は200[mm]で統一している.
本研究において,使用しているサーボモータは,制御周期20[msec]で角度制御を行っている.
また,空気圧人工筋は,外部のエアコンプレッサによって空気を供給しており,最大で0.6[MPa]給気することが可能である.
空気量を調節する弁にはオンオフ弁を使用し,圧力センサを使用することで圧力制御を行っている.
以下に詳細を述べる.

\begin{figure}[htbp]
\begin{center}
 \includegraphics[clip,width=8.0cm]{}
    \caption{ロボットの外観} % title
    \label{robot_outline}
\end{center}
\end{figure}

\subsection{筋配置}
本研究で使用するロボットには,片脚で4本,合計で8本の空気圧人工筋を使用している.
これらの筋は,ヒトの身体における一関節筋の役割と構造を有している.
