2脚ロボットによる歩行や走行の研究は,ロボティクスの分野において数多く行われている.
その手法の1つに,ヒトの運動を簡易モデル化し,それに基づくロボットを作る方法がある.
モデル化によって,ロボットの姿勢安定性を保つのではなく,歩行や走行などの運動の持続性を保つことを目的とすることで,単純な制御則によって歩行や走行の実現を可能としてきた.
しかし,歩行と走行両方を実現可能なモデルは発見されていない.
この理由は,歩行運動を説明するためには,走行運動にはない,両脚支持期における支持脚と遊脚の切り替えを考慮しなければならないことが挙げられる.

本研究では,歩行モデルでの剛体脚と,走行モデルでのばね脚を機構的に切り替え可能なモデルと,その2つの運動の重心軌道に着目した制御則を提案する.
それに伴い,提案するモデルに基づくような2脚ロボットを試作する.


モデルによるシミュレーションの結果,目標水平速度のみを変化させることによって,提案するモデルで歩行と走行両方を達成可能であることがわかった.
また,ロボットによる実験の結果,モデルの歩行における重心軌跡を再現することができたが,走行の実現は達成できなかった.