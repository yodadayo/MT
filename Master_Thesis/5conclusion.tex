本研究では、これまでのモデルでは実現できなかった、歩容の遷移が可能なモデルとその制御則を提案し、それに基づくヒト型二脚ロボットの試作を目的としていた。
われわれはまず、従来の研究で利用されてきた歩行モデルである倒立振子モデルと、走行モデルであるバネマスモデルにおける動作の共通項を調査した。
そこで、剛体脚、ダンパ脚、ばね脚を機構的に切り替えることができるモデルを考案した。
また、脚の自発的動作と受動的動作に着目すると、いずれの歩容においても、重心が描く軌道が一致する点に着目し、仮想支持脚を用いた提案モデルの制御則を使用し、シミュレーションを行った。
さらに、提案したモデルがヒトのような回転軸をもつリンク機構に置き換え、試作したヒト型二脚ロボットによる動作実験を行った。

モデルによるシミュレーションでは、脚モデルの水平方向への目標速度のみを変更可能なパラメータとして、振る舞いの変化を調査した。
実験の結果、目標水平速度を変更するだけで、モデルの歩容を変化させることが可能であることが検証できた。
歩行及び走行中の速度の変化や重心軌道を観察すると、走行時には安定した運動の様子が見られ、歩行時には若干の動作のふらつきが見られたものの、十分な時間が経った場合には、安定して周期的な歩行動作が実現できていることがわかった。

次に、提案した脚モデルをロボットとして実機化するため、2脚ロボットを試作した。
このロボットには、脚構造におけるばね要素とダンパ要素を同時に実現可能な空気圧人工筋を膝関節と距腿関節に使用することで、モデルにおける脚構造の変化や後脚の蹴り出しを実現できるように設計した。
ロボットによる歩行実験を行い、支持脚の状態と重心軌道をモーションキャプチャを用いて調査したところ、モデルのような挙動はあまり実現できていなかった。
また、歩行の実現という点においても、最大3歩までの実現にとどまった。
これらの原因として、ロボットの全長が小さかったために空気圧人工筋の性能を発揮することが難しく、後脚の蹴り出しの力が足りなかったため、ヒトの遊脚の動きを実現できなかった点が挙げられる。
また、走行に関しても同様の理由から、跳躍を行うことができず、動作を実現することが難しかった。
ロボットの筋配置や、今回制御の簡単化の目的から実装しなかった、二関節筋の実装などによって、ロボットの出力不足の問題を緩和できるのではないかと考えている。

本研究におけるモデルの実機化には、改善できる点が多くある。
例えば、試作したロボットでは、モデルにおける重要なパラメータであった水平速度を可視化できるような指標がなかった。
また、ロボットにはほとんどセンサが実装されておらず、外乱や接地などの外部からの信号を検出することはできない。
これらの問題を、ハードウェアやソフトウェアの改善によって解決することで、モデルの動作の再現により近づけることができると考えている。