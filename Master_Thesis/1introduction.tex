近年,ヒトのような二足歩行や二足走行をロボットで再現する研究が行われており,安定な歩行や走行を実現した例が多く存在する.
Narioka et al.は,矢状面,前額面,水平面にロボットの運動を分割し,制御パラメータの探索を行うことで,3次元歩行を実現した\cite{pneumatbt}. % pneumat-bt
Ogawa et al.は,ヒトの3次元的筋配置を模倣したロボットを開発し,実際のヒトの筋電図に基づいた歩行モーションの再現により,3次元歩行を実現した\cite{pneumatbs}. % pneumat-bs % ここまで歩行
Niiyama et al.は,ヒトの下半身の筋骨格構造を模倣したロボットを開発し,ヒトの筋電図に基づきロボットの運動を制御することで,走行を実現した\cite{athleterobot}. % athlete robot % ここから走行
Raibert et al.は,1脚跳躍ロボットの位置制御を2脚ロボットの制御に拡張することで,二足走行を実現した\cite{Raibert:2000:LRB:518526}. % legged robots that balance
これらの研究に共通しているのは,ロボットの姿勢安定性ではなく,運動の持続性を維持することに着目することで,単純な制御則で歩行あるいは走行を実現している点である.

単純な制御則のみを用いてヒトのような運動を実現する手法として,簡略化されたモデルに基づく手法がある.
この手法を用いることで,ヒトの3次元運動を平面に分割し,制御性を向上させることができるという利点がある.
実際に,従来のロボットの研究では,ヒト特有の運動を説明可能なさまざまなモデルが提案されてきた.
その例が,歩行を表現する倒立振子モデル\cite{McGeer:1990:PDW:83528.83533,mobl,doi:10.1152/ajpregu.1977.233.5.R243}や,走行を表現するバネマスモデル\cite{BLICKHAN19891217,doi:10.1098/rspb.2006.3637}である.
これらのモデルを運動の制御に適用することで,ヒトの歩行及び走行の理解や,ロボットによる運動の実現に用いられてきた\cite{doi:10.1163/156855303321165097,893174,Seyfarth2547}.

一方,これまで提案されてきたモデルは,歩行と走行を異なるパラダイムとして扱ってきたため,同一のモデルで歩行と走行両方の運動を実現することは難しい.
そこで,Geyer et al.は,この問題を解決するために,走行の表現に用いられるバネマスモデルを拡張したモデルを考案し,歩行を説明しようとした\cite{doi:10.1098/rspb.2006.3637}.
提案されたモデルは,バネマスモデルに2本目の脚をつけることで,歩行と走行の両方を達成することであることを示した.
しかし,この拡張バネマスモデルでは,固有周波数が運動に大きく影響してしまう.
すなわち、ばね定数の調節によって速度を変化させることは可能であるが,それと同時に,歩行速度や足の接地点,脚の切り替え速度も変化してしまう.
このため,拡張バネマスモデルは,歩行モデルの制御や,従来の歩行モデルを説明するためには用いることはできない.

ヒトの歩行と走行の2つの達成が困難な理由として、従来の歩行モデルを説明するためには,走行動作には含まれない,両脚支持期における支持脚と遊脚の切り替えを考慮する必要があることが挙げられる.
一方で,それぞれの歩容における重心軌跡に着目すると,受動的動作によって生じる下凸の放物線運動と,自発的動作によって行われる上凸の放物線運動を周期的に繰り返す,共通の特徴を有している.
この共通項において,走行における制御則を歩行に拡張することで,2つの運動における脚の動作を同時に説明できる可能性がある.
そこで,本研究では,歩行運動と走行運動における共通項に着目し,進行方向への速度変化によって歩容を変化させることが可能な制御モデルを提案することと,その制御モデルに基づくロボットを試作することを目的とする.

本論文は,以下のような構成となっている.
2章では,目標速度の変化によって,歩行と走行の遷移が実現可能なモデルを提案し,シミュレーションにおける歩容の変化を示す.
3章では,2章で提案したモデルを再現するために試作したロボットの説明を行い,4章でロボットを用いた実験の内容について述べる.
最後に5章で,本研究における結論を述べる.